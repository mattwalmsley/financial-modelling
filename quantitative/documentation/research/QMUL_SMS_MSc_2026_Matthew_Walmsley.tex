\documentclass[12pt, oneside]{book}
\usepackage{graphicx}  % this is for includegraphics

\usepackage{setspace}
\onehalfspacing{} % this sets spacing to 1.5

\usepackage{amsmath}
\usepackage{amsthm} % for theorems, lemmas etc
\usepackage{amsfonts}


\usepackage[colorlinks=true, urlcolor=blue, pdfborder={0 0 0}]{hyperref}
\hypersetup{
     colorlinks   = true,
     citecolor    = blue
}

\theoremstyle{plain}
\newtheorem{theorem}{Theorem}[section]
\newtheorem{proposition}[theorem]{Proposition}
\newtheorem{lemma}[theorem]{Lemma}
\newtheorem{corollary}[theorem]{Corollary}
\newtheorem{fact}[theorem]{Fact}

\theoremstyle{definition}
\newtheorem{definition}[theorem]{Definition}
\newtheorem{example}[theorem]{Example}
\newtheorem{remark}[theorem]{Remark}
\newtheorem{remarks}[theorem]{Remarks}

\newcommand{\Cov}{\mathrm{Cov}}
\newcommand{\Var}{\mathrm{Var}}

\begin{document}

\begin{titlepage}
\begin{center}
        \vspace{-2cm}
        Financial Mathematics MSc Dissertation MTHM038, 2025/27
	\\
        \Huge
        \textbf{Tittle of the Thesis}
        \\        
        \vspace{0.4cm}
        \LARGE
        With special emphasis on examples
        \\        
        \vspace{0.4cm}        
        \textbf{Matthew Walmsley, ID 251012939}   
        \\
        \large Supervisor: Dr Linqi Wang
        \\
        \vspace{0.9cm}
        \includegraphics[scale=0.3]{QMCrest.jpg}
        \\
        \vspace{0.9cm}        
        \LARGE 
        A thesis presented for the degree of\\
        Master of Science in Financial Mathematics\\
        \vspace{0.7cm}        
        \Large
        School of Mathematical Sciences\\ 
        and School of Economics and Finance\\
        Queen Mary University of London \\
    \end{center}
\end{titlepage}


\chapter*{Declaration of original work}
\begin{flushright}
This declaration is made on \today.
\end{flushright}


{\bf Student's Declaration:}
I Matthew Walmsley hereby declare that the work in this thesis 
is my original work. I have not copied from any other students' work, 
work of mine submitted elsewhere,  or from any other sources except where due 
reference or acknowledgement is made explicitly in the text. 
Furthermore, no part of this dissertation has been written for me by another person,
by generative artificial intelligence (AI), or by AI-assisted technologies.  
 



Referenced text has been flagged by:
\begin{enumerate}
\item Using italic fonts, {\bf and} % LaTeX: {\it text}  
\item using quotation marks ``\ldots '', {\bf and}
\item explicitly mentioning the source in the text.
\end{enumerate}

%This excludes any definitions known from your modules or that can be found in an undergraduate text book.

\newpage

\thispagestyle{empty}
        \begin{flushright}
                This work is dedicated to ABC XYZ
        \end{flushright}
\vspace{\stretch{2}}\null%



\chapter*{Acknowledgements}
Example text

\chapter*{Abstract}
Example text

\tableofcontents

\chapter{Introduction}
This note presents a conjecture stemming from our investigations in the generation of sigmoid tensor categories of Picard numbers of tori in Banach algebras.
Example text
 
\section{Motivation for this work}
In the works of Petri (\cite[Theorem 2.3]{Petri}) we find the following statement

\begin{theorem}[{\cite[Theorem 2.3]{Petri}, see also~\cite[pg. 45]{BlackScholes}}]
The Gramm matrix for $E_8$ is:
\[
\begin{pmatrix}
2	&-1&0	&0	&0	&0	&0	&0\\
-1	&2	&-1	&0	&0	&0	&0	&0\\
0	&-1	&2	&-1	&0	&0	&0	&-1\\
0	&0	&-1	&2	&-1	&0	&0	&0\\
0	&0	&0	&-1	&2	&-1	&0	&0\\
0	&0	&0	&0	&-1	&2	&-1	&0\\
0	&0	&0	&0	&0	&-1	&2	&0\\
0	&0	&-1	&0	&0	&0	&0	&2
\end{pmatrix}.
\]
\end{theorem}

\subsection{The problem of exponential extensions}
Example text
\subsection{The approach of Junderstein}
Example text

\chapter{Eulerian topological string motives}
Example text
\section{Definitions}
Example text
\subsection{Tate's theorem}
\paragraph{Preliminary considerations}
Example text
\paragraph{Motivic financial algebroids}
Example text

\subsection{Grothendieck topologies}
Example text

\section{Calculation of the invariant cycles}
Example text

\subsection{Fontaine's theorem}
Example text

\chapter{Conclusions}
Example text

\appendix
\chapter{Implementation of the {\tt BarrierOptionCVA} class}
Example text
\chapter[Shorter running title]{Additional details on the Gundermanian determinant}
Example text



\bibliographystyle{plain}
\bibliography{references}



\end{document}
